\usepackage{multicol} %部分的に2段組みにするパッケージ
\usepackage[dvipdfmx]{graphics} %mac グラフィック周り
%\usepackage[dvipdfmx]{graphicx,color} %windowsの方はこっち
\usepackage{tikz} %描画用
\usepackage{listings} %ソースコード載せるもの
\usepackage{arydshln} % 破線を入れる
\usepackage{amsmath} 
\usepackage{ascmac} %screemで枠表示
\usepackage{tikz} %描画用
\renewcommand{\figurename}{図}
\renewcommand{\tablename}{表}  
\newcommand{\hmbar}{h \kern -0.5em\raise 0.5ex \hbox{--}}%hバー
\newcommand{\lbar}{\lambda \kern -0.5em\raise 0.5ex \hbox{--}} %ラムダバー
\usetikzlibrary{automata}
\usepackage {framed,color}

%答え場合分け
%equationの中で
%\begin{cases}
%hoge \\
%fuga
%\end{cases}
%
%

% 高さの設定
\setlength{\textheight}{\paperheight}   %紙面を本文領域に
\setlength{\topmargin}{-5.4truemm}      %上の余白を20mm(=1inch-5.4mm)に
\addtolength{\topmargin}{-\headheight}   
\addtolength{\topmargin}{-\headsep}     %ヘッダの分だけ本文領域を移動させる
\addtolength{\textheight}{-40truemm}    %下の余白も20mmに
% 幅の設定
\setlength{\textwidth}{\paperwidth}     %紙面を本文領域に
\setlength{\oddsidemargin}{-5.4truemm}  %左の余白を20mm(=1inch-5.4mm)に
\setlength{\evensidemargin}{-5.4truemm} % 
\addtolength{\textwidth}{-40truemm}     %右の余白も20mmに

\setlength{\columnsep}{2zw}  %段間にあける幅 
